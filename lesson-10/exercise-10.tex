%\documentclass{article}
%\documentclass[fleqn]{article} % flush left equation
%\documentclass[leqno]{article} % left equation numbers
\documentclass[fleqn,leqno]{article}
\usepackage[T1]{fontenc}
\usepackage{amsmath}
\usepackage{amsfonts}

\begin{document}

Linear equation:
\[
  y = mx + c
\]

Linear equation: $y = mx + c$.

Integral:
\[
\int_{-\infty}^{+\infty} e^{-x^2} \, dx
\]

Integral:
\begin{equation}
\int_{-\infty}^{+\infty} e^{-x^2} \, dx
\end{equation}

Integral:
\begin{equation*}
\int_{-\infty}^{+\infty} e^{-x^2} \, dx
\end{equation*}

Integral: $\int_{-\infty}^{+\infty} e^{-x^2} \, dx$ % the integration symbol gets significantly reduced in size when using it in-line compared to display mode.

Greek alphabet
\[
\alpha  \beta \gamma \delta \Delta \epsilon \varepsilon \zeta \eta \gamma \Gamma
\]

$\text{text}$
$\mathit{italics}$
$\mathrm{romanized}$
$\mathit{\mathrm{nested text}}$
$\mathrm{\mathit{nested text}}$
$\mathbf{nested text \mathit{works} like this}$
% for nested text, the innermost/local styling applies, probably so allow for different styles within a nested text (like when we want to emphasize something)

\end{document}