\documentclass{article}
\usepackage[T1]{fontenc}
%\newcommand{\diff}{\mathop{}\!d} % for italic
\newcommand{\diff}{\mathop{}\!\mathrm{d}} % for upright

\begin{document}
A sentence with inline mathematics: $y = mx + c$.
A second sentence with inline mathematics: $5^{2}=3^{2}+4^{2}$.

A second paragraph containing display math.
\[
	y = mx + c
\]
See how the paragraph continues after the display.

Superscripts $a^{b}$ and subscripts $a_{b}$.

Some mathematics: $y = 2 \sin \theta^{2}$.

A paragraph about a larger equation
\[
\int_{-\infty}^{+\infty} e^{-x^2} \, dx % the \ before the dx is manual spacing
\]

A paragraph about a larger equation
\[
\int_{-\infty}^{+\infty} e^{-x^2} \, \diff x % using a command for \diff so its easy to adjust
\]

% or use an equation environment, which is numbered
\begin{equation}
\int_{-\infty}^{+\infty} e^{-x^2} \, dx
\end{equation}


% can't have blank lines within the display

% if need several line of math, do not use consecutive display math envs (produces inconsistent spacing), use one of the multi-line display environments such as `align` from the `amsmath` package.
\end{document}