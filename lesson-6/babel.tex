\documentclass{article}
\usepackage[T1]{fontenc}

\usepackage[russian]{babel}
%\usepackage[mongolian]{babel}

%\usepackage[width = 6cm]{geometry} % to force hyphenation here
\usepackage[top=1in, left=0.2in, right=6cm]{geometry}

\usepackage{lipsum}

\begin{document}

This is a lot of filler which is going to demonstrate how LaTeX hyphenates
material, and which will be able to give us at least one hyphenation point.
This is a lot of filler which is going to demonstrate how LaTeX hyphenates
material, and which will be able to give us at least one hyphenation point.

This text consists of other European languages. Я человек больной... Я злой человек. Непривлекательный я человек. Я думаю, что у меня болит печень. Впрочем, я ни шиша не смыслю в моей болезни и не знаю наверно, что у меня болит. Я не лечусь и никогда не лечился, хотя медицину и докторов уважаю. К тому же я еще и суеверен до крайности; ну, хоть настолько, чтоб уважать медицину. (Я достаточно образован, чтоб не быть суеверным, но я суеверен). Нет-с, я не хочу лечиться со злости. Вот этого, наверно, не изволите понимать. Ну-с, а я понимаю. Я, разумеется, не сумею вам объяснить, кому именно я насолю в этом случае моей злостью; я отлично хорошо знаю, что и докторам я никак не смогу "нагадить" тем, что у них не лечусь; я лучше всякого знаю, что всем этим я единственно только себе поврежу и никому больше. Но все-таки, если я не лечусь, так это со злости. Печенка болит, так вот пускай же ее еще крепче болит!

% ^ passage from http://learningrussian.net/notes_from_underground.php

\lipsum

\end{document}